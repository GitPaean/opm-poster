\documentclass[UKenglish,10pt]{beamer}
\usepackage{amsmath, amsfonts, amssymb}
\usepackage{pgf, tikz, bbm, svn}
\usepackage[latin1]{inputenc}
\usepackage{listings}

\lstset{language=C++, basicstyle=\tiny}
\newcommand{\Code}[1]{\texttt{#1}}


\usefonttheme{professionalfonts}

\definecolor{sintefblue}{rgb}{0.0,0.2,0.4}
\definecolor{dr}{rgb}{0.6,0.0,0.0}
\definecolor{dg}{rgb}{0.0,0.3,0.0}
\definecolor{db}{rgb}{0.0,0.0,0.5}
\setbeamercolor{uppercol}{fg=white,bg=sintefblue!80}
\setbeamercolor{lowercol}{fg=orange,bg=black!15}

\newcommand{\COto}{CO\ensuremath{_\mathsf{2}}}
\newcommand{\Tensor}[1]{\ensuremath{\mathsf{#1}}}
\newcommand{\Vector}[1]{\ensuremath{\boldsymbol{#1}}}
\newcommand{\Matrix}[1]{\ensuremath{\boldsymbol{#1}}}
\newcommand{\T}        {\ensuremath{\mathsf{T}}}
\newcommand{\Grad}     {\ensuremath{\nabla}}

\newcommand{\dunemod}[1]{\textsl{dune-{#1}}}
\newcommand{\opmmod} [1]{\textsl{opm{#1}}}
\newcommand{\dumux}    {DuMu\ensuremath{^{\mathrm X}}}

\def\bfK{{\bf K}}
\def\bfv{{\bf v}}
\def\bfn{{\bf n}}


\DeclareMathOperator{\Div}{div}




\usepackage{listings}
\usepackage{color}
\usepackage{textcomp}
\definecolor{listinggray}{gray}{0.9}
\definecolor{lbcolor}{rgb}{0.95,0.95,0.95}
\definecolor{lightgray}{gray}{0.3}
\lstset{%
  xleftmargin= 4pt,
  xrightmargin=4pt}
\lstdefinelanguage{MRST}{%
  alsoletter={...},%
  morekeywords={%                             % keywords
  break,case,catch,continue,elseif,else,end,for,function,global,%
  if,otherwise,persistent,return,switch,try,while,...},%
  comment=[l]\%,%                             % comments
  morecomment=[l]...,%                        % comments
  morestring=[m]',%                           % strings
}[keywords,comments,strings]%
\lstset{
  backgroundcolor=\color{lbcolor},
  tabsize=4,
  rulecolor=,
  language=MRST,
  basicstyle=\small,
  upquote=true,
  aboveskip={0.5\baselineskip},
  columns=flexible,
  showstringspaces=false,
  extendedchars=true,
%  breaklines=true,
  prebreak = \raisebox{0ex}[0ex][0ex]{\ensuremath{\hookleftarrow}},
  frame=single,
  showtabs=false,
  showspaces=false,
  showstringspaces=false,
  identifierstyle=\ttfamily,
  keywordstyle=\color[rgb]{0,0,1},
  commentstyle=\color[rgb]{0.133,0.545,0.133},
  stringstyle=\color[rgb]{0.627,0.126,0.941},
}
\newcommand{\mcode}[1]{\lstinline|#1|}

\tikzset{Workflow Stage/.style=%
        {rectangle, draw=blue!50, fill=blue!20, thick, %
         text width=35mm}}


\title{Community}
\author[Atgeirr F. Rasmussen]{Atgeirr Fl{\o} Rasmussen}
\institute[SINTEF]{SINTEF ICT, Dept. Applied Mathematics}
\date[2017--01--26]{26th January 2017}

%%%%%%%%%%%%%%%%%%%%%%%%%%%%%%%%%%%%%%%%%%%%%%%%%%%%%%%%%%%%%%%%%%%%%%%%%%%%%%

\begin{document}


%============================================
\section{The OPM Community}
%============================================



% ---------------------------------------------------------------------------
\begin{frame}
  \frametitle{Why Open Source?}
  \begin{block}{Motivation for the industry}
    \begin{itemize}
    \item accelerate technology transfer from academic institutions
    \item new research results available as free software 
    \item accelerate testing of new models and computational methods
    \item limited research budgets -- cannot fund duplication of effort
    \item can attack problems commercial providers ignore
    \end{itemize}
  \end{block}
  \begin{block}{Motivation for academia}
    \begin{itemize}
    \item facilitate testing of new ideas on models with industry complexity
    \item control over discretizations and solvers
    \item ideal for benchmarking, exchange of methods, etc
    \item with a set of basic building blocks one can go right to the
      interesting problems
    \end{itemize}
  \end{block}
  A step in the right direction for reproducible sciences
\end{frame}


% ---------------------------------------------------------------------------
\begin{frame}
  \frametitle{Licensing and copyright}
  \begin{block}{GNU General Public License, version 3}
    \begin{itemize}
    \item one of the most-used software licenses
    \item no restriction on usage 
    \item no restriction on modification
    \item restriction on distribution: must provide recipient with same rights
    \end{itemize}
  \end{block}
  \begin{block}{Copyright on OPM parts}
    \begin{itemize}
    \item mostly held by author organization
    \item the most important parts have multiple authors
    \item copyright holders can relicense under arbitrary terms, but all
      must agree (in general)
    \end{itemize}
  \end{block}
  This means: proprietary programs cannot include OPM code unless {\em all
  contributors agree}. \\
  $\implies$ Building a cooperating community, not just a product.
\end{frame}

% ---------------------------------------------------------------------------
\begin{frame}
  \frametitle{How to learn more}
  \begin{block}{OPM website}
    \begin{itemize}
    \item http://opm-project.org
    \item installation instructions 
    \item usage tutorials
    \item recent news and events
    \end{itemize}
  \end{block}
  \begin{block}{Source code on GitHub}
    \begin{itemize}
    \item http://github.com/OPM
    \item one repository per module
    \item open development through pull requests
    \item issue trackers for bugs and requests
    \end{itemize}
  \end{block}
  \begin{block}{Other means of communication}
    \begin{itemize}
    \item mailing list (go to website to join)
    \item open OPM meetings for developers and users
    \end{itemize}
  \end{block}
\end{frame}



\end{document}